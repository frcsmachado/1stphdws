\section[Introduction]{Introduction}
%%%%%%%%%%%%%%%%%%%%%%%%%%%%%%%%%%%%%%%%%%%%%%%%%%%%%%%%%%%%%%%%%%%%%%%%%%%%%%%%%%%%%%%%%
%
%\subsection{Microgrids}
\begin{frame}{Microgrids Definition:}
\setbeamercovered{transparent}
\begin{block}{\it{European Projects: Microgrids and More Microgrids}}
\begin{itemize}
\item LV Distribution System with Distributed Energy Resources (DER)
\item Storage Devices and Flexible Loads
\item<1-| alert@1> Nonautonomous way if Connected to a Main Grid ($\sf {f_{grid}}$, $\sf{V_{grid}}$)
\item<2-| alert@2> Independent way if Disconnected from the Main Grid ($\sf {f_{\mu Grid}}$, $\sf{V_{\mu Grid}}$)
\end{itemize}
\end{block}

\begin{columns}
\begin{column}{0.5\textwidth}
\includegraphics [width=1\textwidth]{pdf/basic_connected_microgrid.pdf}
\end{column}
\begin{column}{0.5\textwidth}
\includegraphics<2> [width=1\textwidth]{pdf/basic_unconnected_microgrid.pdf}
\end{column}
\end{columns}

\end{frame}
%According to the Eropean Projects: Microgrids and More Microgrids the definition of these comprise LV distribution networks with DER together with Storage devices and Flexible loads which can be operated in a nonautonomous way when they are connected to a main grid where the voltage and frecuency of microgrid is imposed by the main grid, or they can work  independently  when they are disconnected from a main grid and voltage and frequency must be esbablished by itself.

\begin{frame}{Fundamental Key of Microgrids}
\begin{block}{Coordinated Management of 
DERs, Storage and Loads}
\begin{itemize}
\item<1-| alert@1> Increasing the Efficiency, Facilitating DERs integration and Flexibility
\item Collect Data from $\mu$Grid: P, Q, I, V, f, $\theta$, etc...
\item Control (Central or Decentral.). Set Points: DERs, Storage, Loads
\end{itemize}
\end{block}

\begin{columns}
\begin{column}{0.5\textwidth}
\includegraphics [width=1\textwidth]{pdf/managed_connected_microgrid.pdf}
\end{column}
\begin{column}{0.5\textwidth}
\includegraphics [width=1\textwidth]{pdf/managed_unconnected_microgrid.pdf}
\end{column}
\end{columns}

\end{frame}

%However, the fundamental key of microgrids is to manage all this elements in a coordinated way in order to increase the efficiency of this system, facilitating the integration of DERs and increasing the flexibility of operation of the system. For this it is necessary to develop a communication structure to get data  such as active and reactive power, current and voltage, frequency or angle etc ...from the different elements of the microgrids that can be used to design a controller, centralized or decentralized, to achieve the previous objectives by defining the  set point of DERs, storage and loads.

%\subsection{Distribution Networks with High Penetration of DERs}
\begin{frame}{Distribution Network with High Penetration of DERs}
%\setbeamercovered{transparent}
\begin{columns}
	\begin{column}{0.5\textwidth}
		\setbeamercovered{transparent}
		\begin{block}{LV Distribution Networks}
%			\setbeamercovered{transparent}
			\begin{itemize}
			\item Toplogy $\approx$ $\mu$Grids
			\item Radial Operation
			\item DERs, Storage and Flex. Loads
%			\item Control: Secondary Subst.
%			\item<2-| alert@2> Voltage drop
%			\end{itemize}
%		\end{block}
%		\setbeamercovered{transparent}
%		\begin{block}<2>{LV Distribution Networks:}
%			\setbeamercovered{transparent}
%			\begin{itemize}

			\item<2-| alert@2> Reverse Power Flow
			\item<2-| alert@2> Overvoltages and Overloads
			\end{itemize}
		\end{block}
	\end{column}	
	\begin{column}{0.5\textwidth}
	 \begin{center}
	 		\includegraphics<1> [width=1\textwidth]{pdf/distribution_network_radial.pdf}
%	 		\vspace{80pt}
	 		\includegraphics<2> [width=1\textwidth]{pdf/distribution_network_radial_estado.pdf}
	 		%\includegraphics<3> [width=1\textwidth]{pdf/actuall_distribution_network_v3.pdf}
	 		%\includegraphics<4> [width=1\textwidth]{pdf/actuall_distribution_network_v3a.pdf}
	 \end{center}
	\end{column}
\end{columns}
\end{frame}

%In the case of LV Distribtuon Networks, the topology of the grid is almost the same than microgrids. These are usually operated in radial configuration, although they can be reconfigurated by closing the breaker in the switching center to mesh the network. Moreover, DERs, storage systems and flexible loads as EV can be also connected to the  distribtuion newtwork providing to them the same resources as microgrids. In case of a high power injection of DERs reverse power flows to the secondary substations could occur causing overvoltages and overloads in feeders and transformers that could damage the electrical devices or reducing its useful life icreasing the investment in the electrical network

%\subsection{DC links in Distribution Networks}
\begin{frame}{DC links based on VSCs}
\begin{columns}
%	\vspace{20pt}
	\column[b]{0.5\textwidth}
	\begin{block}{Controlled Power Flow:}
	\setbeamercovered{transparent}
	\begin{itemize}
	\item<1-| alert@1> $\Downarrow{}$Losses 
	\item<1-| alert@1> $\Uparrow{}$Loadability
	\item<1-| alert@1> $\Uparrow{}$DER penetration 
	\item<2-| alert@2>  Improving voltage profile 
%	\item<2-| alert@2>  Voltage sag mitigation
	\item<2-| alert@2>  Dischar. feeders/transformers
%    \item<2-| alert@2>  $\Downarrow{}$ Harmonics and unbalances
	\end{itemize}
	\end{block}
	\vspace{40pt}
	\column[b]{0.5\textwidth}
	\includegraphics<1> [width=1\textwidth]{pdf/distribution_network_DC_link.pdf}
	\includegraphics<2> [width=1\textwidth]{pdf/distribution_network_DC_link_estado.pdf}
\end{columns}
\end{frame}

%One of the proposals to solve this problem is to replace the switching center by a flexible DC links based on VSCs in order to interconnect adjacent radial feeders. In this way, active and reactive power flows can be properly controlled facilitating high penetration of DERs and improving the efficiency of the system without additional investment in the electrical network. In addition, among others, the controllable power flows can be established to maintain voltages and currents within the techical limits.